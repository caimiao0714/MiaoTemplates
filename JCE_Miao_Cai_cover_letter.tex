\documentclass[10pt]{article}
\usepackage[a4paper, margin=40mm]{geometry}
\usepackage[breaklinks=true]{hyperref}
\usepackage{enumerate}
\hypersetup{colorlinks,urlcolor=blue}
\pagenumbering{gobble}
\setlength{\parskip}{0.5em}%
\setlength{\parindent}{0pt}%

\begin{document}

\noindent

\today\\
Drs. J Andr{\'e} Knottnerus and Peter Tugwell\\
Editors-in-Chief\\
Journal of Clinical Epidemiology

\vspace*{\fill}

Dear Drs. Knottnerus and Tugwell,\\

We are pleased to submit an original research article entitled ``Association between maternal physical exercise and the risk of preterm birth: A case-control study in Wuhan, China" to \textit{Journal of Clinical Epidemiology} for review and consideration.

In this manuscript, we conducted a large case-control study that includes 2,396 pregnant women who experience preterm birth and 4,263 controls in a large central Chinese city. We attempt to examine the association between physical exercise during pregnancy and preterm birth. Many researchers investigated this association using categorical measures of physical exercise. By contrast, we used a quantitative measure of physical exercise (the number of minutes per day), and used a Bayesian generalized additive mixed model to explore the nonlinear relationship. We have the following major findings:

\begin{enumerate}[(A)]
	\item Bayesian generalized additive mixed model (GAMM) showed a ``U-shaped" relationship between physical
	exercise and pre-term birth risk.
	\item The probability of experiencing PTB decreased as physical exercise increased from zero to about 150 minutes per day. However, the association became positive once physical exercise exceeded 150 minutes per day.
\end{enumerate}

This study suggests that physical exercise during pregnancy, at a moderate amount and intensity, is associated with a lower risk of PTB. Pregnant women are not recommended to be engaged with high participation in physical exercise (150 minutes per day). Therefore, we believe that this study represents a valuable addition to the existing literature.

We affirm that this manuscript has not been published elsewhere and is not under consideration by another journal. The corresponding author of this manuscript will be Bin Zhang, M.D. at \href{mailto:2500070378@qq.com}{2500070378@qq.com} and Zhengmin Qian, Ph.D., M.D. at \href{mailto:zhengmin.qian@slu.edu}{zhengmin.qian@slu.edu}. Thank you for considering our
manuscript.





\vspace*{\fill}


On behalf of the research team,\\
%\vspace*{20pt}


Zhengmin Qian, M.D., Ph.D.\\
Professor and Chair\\
Department of Epidemiology and Biostatistics\\
College for Public Health and Social Justice\\
Saint Louis University\\
3545 Lafayette Avenue\\
St. Louis, MO 63108, USA\\
Email: \href{mailto:zhengmin.qian@slu.edu}{zhengmin.qian@slu.edu}\\
Tel.: 1-314-977-8158



\end{document} 