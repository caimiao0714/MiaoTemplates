\documentclass[11pt, a4paper]{article}
%\usepackage[a4paper, margin=30mm]{geometry}
\usepackage[breaklinks=true]{hyperref}
\usepackage{enumerate}
\hypersetup{colorlinks,urlcolor=blue}
\usepackage{fontspec} 
\usepackage[UTF8]{ctex}
\usepackage{epigraph}
\usepackage{titlesec}
\usepackage{graphicx,wrapfig,lipsum}
% DOCUMENT LAYOUT
\usepackage{geometry} 
\geometry{a4paper, left=25mm, right=25mm, top=10mm, bottom=20mm}
\setlength\parindent{0in}
\usepackage[export]{adjustbox}

% FONTS
\usepackage[usenames,dvipsnames]{xcolor}
\usepackage{xunicode}
\usepackage{xltxtra}
\defaultfontfeatures{Mapping=tex-text}
\setmainfont[
   Ligatures={Common}, Numbers={OldStyle}, Variant=01,
  %BoldFont=\fzxbs,
  BoldFont={LinLibertine_RB.otf},
  ItalicFont=LinLibertine_RI.otf,
  BoldItalicFont=LinLibertine_RBI.otf
]{LinLibertine_R.otf}
\setmonofont[Scale=0.8]{DejaVuSansMono.ttf}

\definecolor{clemsonpurple}{HTML}{522D80}
\definecolor{clemsonorange}{HTML}{F66733}
\definecolor{uiucblue}{HTML}{003C7D}
\definecolor{uiucorange}{HTML}{F47F24}
\definecolor{slublue}{HTML}{003EA5}

% ---- CUSTOM COMMANDS
\chardef\&="E050
\newcommand{\html}[1]{\href{#1}{\scriptsize\textsc{[html]}}}
\newcommand{\pdf}[1]{\href{#1}{\scriptsize\textsc{[pdf]}}}
\newcommand{\doi}[1]{\href{#1}{\scriptsize\textsc{[doi]}}}
% ---- MARGIN YEARS
\usepackage{marginnote}
\newcommand{\amper{}}{\chardef\amper="E0BD }
\newcommand{\years}[1]{\marginnote{\scriptsize #1}}
\renewcommand*{\raggedleftmarginnote}{}
\setlength{\marginparsep}{7pt}
\reversemarginpar

% HEADINGS
\usepackage{sectsty} 
\usepackage[normalem]{ulem}
% SET FONTS
\setCJKfamilyfont{MyCJKfzxbs}{FZXiaoBiaoSong-B05S} %方正小标宋简体
\newcommand{\fzxbs}{\CJKfamily{MyCJKfzxbs}}

\sectionfont{\fzxbs\bfseries\upshape\Large}
\subsectionfont{\fzxbs\bfseries\scshape\normalsize} 
\subsubsectionfont{\fzxbs\bfseries\upshape\large} 



\pagenumbering{gobble}
\setlength{\parskip}{0.8em}%
%\setlength{\parindent}{0pt}%

\begin{document}
\begin{minipage}[c]{0.5\linewidth}
\textcolor{slublue}{Department of Epidemiology and Biostatistics}\\
\textcolor{slublue}{College for Public Health \& Social Justice}\\
\textcolor{slublue}{Saint Louis University}\\
\textcolor{slublue}{3545 Lafayette Ave., St. Louis, MO 63103}
\end{minipage}
\hfill
\begin{minipage}[c]{0.25\linewidth}
\includegraphics[trim=25 35 25 25, clip, width=\linewidth, right]{SLU.png}%\frame{}
\end{minipage}

\vspace{-5pt}
\par\noindent\rule{\textwidth}{0.4pt}
\vspace{-28pt}
%\noindent

\today\\
尊敬的南方科技大学的领导和老师,

\vspace*{\fill}

\setlength{\parindent}{24pt}%
你们好!我是美国退伍军人事务部(Department of Veteran Affairs)临床流行病学研究中心圣路易斯分部的生物统计研究员(biostatistician)蔡苗。研究方向包括贝叶斯统计、临床流行病学、环境流行病学、交通运输安全等。我从南方科技大学的官方网站上看到了此次2021年南方科技大学国际交叉学科论坛医学院的通知,在简单浏览了学校的招聘岗位和要求之后,我认为我符合优秀青年人才或优秀博士后的要求,因此在此申请本次论坛和应聘学校相关岗位。

在本硕博的学习过程中,我一直成绩优异。本科在华中科技大学求学过程中,加权平均分位列全班第二,获得推荐免试研究生的学习资格;在美国圣路易斯大学的学习期间,我拿到了学院的博士全额奖学金资助,在60个学分的课程项目中拿到了GPA 3.95/4.0的成绩。优异的成绩也为我赢得了一次国家奖学金,两次国家励志奖学金,一次香港道德会奖学金,两次国家留学基金委超过6万美金的生活费资助,以及美国国家自然科学基金6万美金的研究助理资助。

在学术研究方面,我参与过两个中国国家自然科学基金面上项目,数十次中国省部级科研项目,美国两个国家级研究项目和一次基金撰写。我在美国退伍军人事务部临床流行病学研究中心有接近两年的全职研究经历,与各个学科的研究人员有着密切且深入的讨论、日常会议、以及研究合作经验。在JAMA Network Open, Transportation Research Part C: Emerging Technologies等杂志上发表\textbf{\textcolor{slublue}{第一作者SCI/SSCI英文论文7篇}},合作论文共21篇。在中文核心期刊上发表论文12篇,其中第一作者文章5篇。相较于其它应聘者,我认为我有以下突出优势:
\vspace{-0.3\baselineskip}
\begin{enumerate}[(A)]
	\setlength{\parskip}{-0.2em}%
	\item \textbf{研究兴趣与合作广泛}:在中美两国以及多个学科(统计、经济、临床流行病、交通运输等)都建立了广泛的研究合作和学术联系,并且熟悉中美两国的学术研究开展方式和特点,有助于学校开展国际合作和跨学科交叉研究。
	\item \textbf{统计理论与实践结合}:我的博士主要研究方向是贝叶斯统计在大数据中的应用,因此掌握统计相关理论,并且能够运用R/Python/SAS/SQL等编程语言实现大数据的建模统计分析,这是未来特别有前景的研究方向。
	\item \textbf{发展潜力和空间巨大}:我的未来研究开展方向为依托于电子病历信息的经济统计建模研究。在中国老龄化和信息系速度越来越快的背景下,对医疗的需求与日俱增,并且医疗电子信息系统越来越完善,我坚信这个方向能够为国家和社会带来巨大价值。
\end{enumerate}



感谢南方科技大学的厚爱与支持!期待能有机会与大家在深圳合作和研究。



\vspace*{\fill}


\setlength{\parindent}{0pt}%
\begin{minipage}[c]{0.7\linewidth}
\includegraphics[trim=0 10 0 10, clip, width=0.2\linewidth]{Miao_Cai_sig_CN}\\
蔡苗\\
美国圣路易斯大学\\
公共卫生与社会正义学院流行病与生物统计系\\
电子邮箱:\href{mailto:miao.cai@outlook.com}{miao.cai@outlook.com}
\end{minipage}
\hfill
\begin{minipage}[c]{0.3\linewidth}
%\includegraphics[trim=0 160 0 160, clip, width=\linewidth]{va.pdf}
\end{minipage}

\end{document} 